\documentclass{article}
\begin{document}
\title{How to Learn Code{}}
\author{OBURUSULE DUSTAN}
\maketitle
\begin{abstract}
As a computer scientist I have been introduced to many programming languages at my time with the Computer Science course. I have compiled some few facts about how to code according to my experiences and those of others that are also in the field that involves computer programming. Now teaching yourself how to code is very difficult, the total of resources out there is barely overwhelming, you have very many resources as soon as you go online and search about code. For example, code academy, edx etc.
But the problem you face is trying to narrow done your way of learning for example a situation where you want to learn this and that and there so many things to learn. And at this point I want to include the different ways that you can learn code.
\end{abstract}
\subsection{ Some of the common problems that exist}
\begin{description}
\item[1.	Boring Courses. ]
\begin{itemize}
 \item[]  
 \item 
Almost all the other courses out there online are very boring. They are made by somebody who is sitting behind the screen, like a screen cast and they are like sitting there and talking about something and don’t treat you like you’re also a human being. For example, the host isn’t even talking with interest. You want somebody to talk to you, guide you and make you go through that process because specially when you’re starting out that’s what you need. So not hiding back in a screen cast and speaking to you in a boring. Some that’s one common problem with a lot of these courses. 
 \end{itemize}
\item[2.	Not Really for beginners]
\begin{itemize}
 \item[] 
\item
These courses don’t centre around beginners, a lot of these courses will be called one on one courses but they don’t scale gradually the scale is somehow broken, just like a stepwise mathematics which means the lines aren’t actually connected or exponential difficulty scale. So they say they centre around beginners but they take a lot of things for granted and skip over a lot of little steps which leave you frustrated and sometimes you think that your dump. But they should teach you those things because there is no other way that you would have learnt them and they shouldn’t assume that especially when you’re starting out. And that’s another problem taking things for granted.
\end{itemize}
\item[3.	Slow Feedback]
\begin{itemize}
\item []
\item
Feedback is slow, so in most of these courses that you’re taking you get stuck in a problem or you want to try out some new idea that you came up with but where do you get feedback from for it. Well the best solution you come up with after a lot of contemplation is maybe I will just post it in a forum, so then you start thinking about an intelligent way to put together the question so that you would post it in the forums and before you know it a half an hour has passed while you’re trying to compose this brilliant question cause after all you don’t want those really intelligent people on the internet laughing at you. So you do the next best thing that this is taking me so long so what was I thinking of posting the question in the forums and you give up, ultimately you skip that part of the course and you move on. Then you miss out on a fundamental piece of understanding.
\end{itemize}
\end{description}  
\subsection{Solution I attained for the above problems}
After more and more research online, I landed on a website called cleverprogrammer.com. It had a lot of features that could solve the above problems and help me learn coding quickly and efficiently. 
\begin{itemize}
\item It centred around beginners where the host actually starts teaching you individually and use a tool called turtle which is a module and it actually builds code gradually and graphically as we learn code. So this way if what you have as your output isn’t what the host has done, you automatically know that your done is wrong.
\item The feedback is extremely fast, there is a feature that the student uses to communicate directly to the teacher at real time. It’s like a mentor being alongside you, so you learn what you need when you need it
\end{itemize}
\subsection{ Conclusion}
So learning how to code really requires the best resources for you to learn well. Some of the other websites that are good for learning coding depending on your field of interest might include Free code camp for web development, Coursera is also a good resource, udasity has very high quality production resources, edx has intense courses
\end{document}