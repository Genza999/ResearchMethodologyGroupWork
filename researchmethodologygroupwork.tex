\documentclass{article}
\begin{document}
\title{A Conceptual Model for Assessing Privacy Risk}
\maketitle
\section{Introduction}
Privacy, in its broadest sense, is about the right of an individual to be let alone. It can take two main forms, and these can be subject to different types of intrusion: 
\begin{itemize}
\item Physical privacy - the ability of a person to maintain their own physical space or solitude. Intrusion can come in the form of unwelcome searches of a person’s home or personal possessions, bodily searches or other interference, acts of surveillance and the taking of biometric information 
\item Informational privacy – the ability of a person to control, edit, manage and delete information about themselves and to decide how and to what extent such information is communicated to others. Intrusion can come in the form of collection of excessive personal information, disclosure of personal information without consent and misuse of such information. It can include the collection of information through the surveillance or monitoring of how people act in public or private spaces and through the monitoring of communications whether by post, phone or online and extends to monitoring the records of senders and recipients as well as the content of messages This code is concerned primarily with informational privacy, but an organisation can use PIAs to assess what they think are the most relevant aspects of privacy.
\end{itemize}
Privacy risk is the risk of harm arising through an intrusion into privacy. This code is concerned primarily with minimising the risk of informational privacy - the risk of harm through use or misuse of personal information. Some of the ways this risk can arise is through personal information being: 
\begin{itemize}
\item inaccurate, insufficient or out of date; 
\item excessive or irrelevant; 
\item  kept for too long;
\item disclosed to those who the person it is about does not want to have it;
\item used in ways that are unacceptable to or unexpected by the person it is about; or 
\item not kept securely. 
\end{itemize}
Harm can present itself in different ways. Sometimes it will be tangible and quantifiable, for example financial loss or losing a job. At other times it will be less defined, for example damage to personal relationships and social standing arising from disclosure of confidential or sensitive information. Sometimes harm might still be real even if it is not obvious, for example the fear of identity theft that comes from knowing that the security of information could be compromised. There is also harm which goes beyond the immediate impact on individuals. The harm arising from use of personal information may be imperceptible or inconsequential to individuals, but cumulative and substantial in its impact on society. It might for example contribute to a loss of personal autonomy or dignity or exacerbate fears of excessive surveillance. The outcome of a PIA should be a minimisation of privacy risk. An organisation will need to develop an understanding of how it will approach the broad topics of privacy and privacy risks. There is not a single set of features which will be relevant to all organisations and all types of project - a central government department planning a national crime prevention strategy will have a different set of issues to consider to an app developer programming a game which collects some information about users. Understanding privacy risk in this context does though require an understanding of the relationship between an individual and an organisation. Factors that can have a bearing on this include: 
\begin{itemize}
\item Reasonable expectations of how the activity of individuals will be monitored.
\item Reasonable expectations of the level of interaction between an individual and an organisation.
\item The level of understanding of how and why particular decisions are made about people. Public bodies need to be aware of their obligations under the Human Rights Act. Article 8 of the European Convention on Human Rights guarantees a right to respect for private life which can only be interfered with when it is necessary to meet a legitimate social need. Organisations which are subject to the Human Rights Act can use a PIA to help ensure that any actions that interfere with the right to private life are necessary and proportionate.
\end{itemize}

\section{Purpose}
Each organisation can decide who is best placed to coordinate and carry out the PIA process. Large organisations are more likely to have a dedicated data protection officer. A DPO is naturally well placed to have a significant role in a PIA, and may also be able to design a set of PIA tools which mirror an organisation’s existing processes. They may also maintain a log of all PIAs carried out in the organisation. Not all organisations have their own DPO, or it may be difficult for a DPO to conduct all the PIAs - the general approach to PIAs is also intended for use by non-experts. Project, risk or other managers without specialist data protection knowledge should be able to use the screening questions in Annex One to help them focus on privacy issues. An effective PIA will include some involvement from various people in an organisation, who will each be able to identify different privacy risks and solutions. Small organisations can still benefit from implementing in some form the core steps of the PIA process.
\section{Project description}
The PIA process is a flexible one, and it can be integrated with an organisation’s existing approach to managing projects. This guidance identifies the key principles of PIAs which the ICO suggests should be included. The implementation of the core PIA principles should be proportionate to the nature of the organisation carrying out the assessment and the nature of the project. The early stages of the PIA will help an organisation to understand the potential impact on privacy and the steps which may be required to identify and reduce the associated risks. This in turn will indicate the level of resources and time which may need to be dedicated to the assessment. The process is designed to be scaled in size depending on the nature of the project. Small projects with a relatively low impact will require a less formal and intensive exercise. Many organisations will benefit from producing their own PIA process and accompanying guidance. This is often the most effective way of ensuring that privacy issues are considered as part of the existing project or risk management procedures and that the PIA covers any sectoral or organisational specific angles. Annex four explains how organisations can integrate PIAs with project and risk management. The process of conducting a PIA should begin early in the project. When it becomes clear that a project will have some impact on privacy an organisation should start to consider how they will approach this. This does not mean that a formal PIA must be started and finished before a project can progress further. The PIA should run alongside the project development process. What begins as a more informal early consideration of privacy issues can be developed into part of the PIA. Annex two provides a template which organisations can use to record the results of each of the steps detailed below. Organisations do not have to use the template and can chose to use existing records management systems or project management tools if they prefer.
When an organisation is planning to introduce PIAs to their projects, there are some important steps at each stage which they should seek to use. This overview also shows some examples of how an organisation can demonstrate that it is following best practice in its PIAs(shown by a !) :
\begin{itemize}
\item • Identifying the need for a PIA
\item • Answer screening questions to identify a proposal’s potential impact on privacy.
\item • Begin to think about how project management activity can address privacy issues.
\item • Start discussing privacy issues with stakeholders.
\item ! Conducted early during the project planning stage.
\item ! The overall aims of the project are described.
\item ! The project development process is adapted to address privacy concerns describing information flows
\item • Explain how information will be obtained, used, and retained – there may be several options to consider. This step can be based on, or form part of, a wider project plan.
\item • This process can help to identify potential ‘function creep’ - unforeseen or unintended uses of the data (for example data sharing).
\item ! People who will be using the information are consulted on the practical implications.
\item ! Potential future uses of information are identified, even if they are not immediately necessary.
\item • Identifying privacy and related risks. Record the risks to individuals, including possible intrusions on privacy where appropriate.
\item • Assess the corporate risks, including regulatory action, reputational damage, and loss of public trust.
\item • Conduct a compliance check against the Data Protection Act and other relevant legislation.
\item • Maintain a record of the identified risks. 
\item ! The process helps an organisation to understand the likelihood and severity of privacy risks. 
\item ! An organisation is open with itself about risks and potential changes to a project.
\item •  Identifying and evaluating privacy solutions. Devise ways to reduce or eliminate privacy risks.
\item • Assess the costs and benefits of each approach, looking at the impact on privacy and the effect on the project outcomes.
\item • Refer back to the privacy risk register until satisfied with the overall privacy impact.
\item ! The process takes into account the aims of the project and the impact on privacy. 
\item ! The process also records privacy risks which have been accepted as necessary for the project to continue.  
\item •Signing off and recording the PIA outcomes. Obtain appropriate signoff within the organisation. 
\item • Produce a PIA report, drawing on material produced earlier during the PIA.
\item • Consider publishing the report or other relevant information about the process.
\item ! The PIA is approved at a level appropriate to the project. 
\item ! A PIA report or summary is made available to the appropriate stakeholders.
\item • Integrating the PIA outcomes back into the project plan
\item • Ensure that the steps recommended by the PIA are implemented.
\item • Continue to use the PIA throughout the project lifecycle when appropriate. 
\item ! The implementation of privacy solutions is carried out and recorded.
\item ! The PIA is referred to if the project is reviewed or expanded in the future.
\end{itemize}
\end{document}