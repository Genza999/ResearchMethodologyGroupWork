\documentclass{article}
\begin{document}
\title{A Conceptual Model for Assessing Privacy Risk}
\maketitle
\section{Introduction}
Privacy, in its broadest sense, is about the right of an individual to be let alone. It can take two main forms, and these can be subject to different types of intrusion: 
\begin{itemize}
\item Physical privacy - the ability of a person to maintain their own physical space or solitude. Intrusion can come in the form of unwelcome searches of a person’s home or personal possessions, bodily searches or other interference, acts of surveillance and the taking of biometric information 
\item Informational privacy – the ability of a person to control, edit, manage and delete information about themselves and to decide how and to what extent such information is communicated to others. Intrusion can come in the form of collection of excessive personal information, disclosure of personal information without consent and misuse of such information. It can include the collection of information through the surveillance or monitoring of how people act in public or private spaces and through the monitoring of communications whether by post, phone or online and extends to monitoring the records of senders and recipients as well as the content of messages This code is concerned primarily with informational privacy, but an organisation can use PIAs to assess what they think are the most relevant aspects of privacy.
\end{itemize}
Privacy risk is the risk of harm arising through an intrusion into privacy. This code is concerned primarily with minimising the risk of informational privacy - the risk of harm through use or misuse of personal information. Some of the ways this risk can arise is through personal information being: 
\begin{itemize}
\item inaccurate, insufficient or out of date; 
\item excessive or irrelevant; 
\item  kept for too long;
\item disclosed to those who the person it is about does not want to have it;
\item used in ways that are unacceptable to or unexpected by the person it is about; or 
\item not kept securely. 
\end{itemize}
Harm can present itself in different ways. Sometimes it will be tangible and quantifiable, for example financial loss or losing a job. At other times it will be less defined, for example damage to personal relationships and social standing arising from disclosure of confidential or sensitive information. Sometimes harm might still be real even if it is not obvious, for example the fear of identity theft that comes from knowing that the security of information could be compromised. There is also harm which goes beyond the immediate impact on individuals. The harm arising from use of personal information may be imperceptible or inconsequential to individuals, but cumulative and substantial in its impact on society. It might for example contribute to a loss of personal autonomy or dignity or exacerbate fears of excessive surveillance. The outcome of a PIA should be a minimisation of privacy risk. An organisation will need to develop an understanding of how it will approach the broad topics of privacy and privacy risks. There is not a single set of features which will be relevant to all organisations and all types of project - a central government department planning a national crime prevention strategy will have a different set of issues to consider to an app developer programming a game which collects some information about users. Understanding privacy risk in this context does though require an understanding of the relationship between an individual and an organisation. Factors that can have a bearing on this include: 
\begin{itemize}
\item Reasonable expectations of how the activity of individuals will be monitored.
\item Reasonable expectations of the level of interaction between an individual and an organisation.
\item The level of understanding of how and why particular decisions are made about people. Public bodies need to be aware of their obligations under the Human Rights Act. Article 8 of the European Convention on Human Rights guarantees a right to respect for private life which can only be interfered with when it is necessary to meet a legitimate social need. Organisations which are subject to the Human Rights Act can use a PIA to help ensure that any actions that interfere with the right to private life are necessary and proportionate.
\end{itemize}
\end{document}